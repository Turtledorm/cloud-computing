\section{Modelo de implantação}

No modelo de implantação, dependemos das necessidades das aplicações que serão
implementadas. A restrição ou abertura de acesso depende do processo de negócios,
do tipo de informação e do nível de visão desejado. Por exemplo, certas
organizações não querem que todos os usuários possam acessar e utilizar
determinados recursos no seu ambiente de computação em nuvem. Seguem abaixo
diferentes tipos de implantação:

\subsection{Nuvem privada}

Uma nuvem privada é de propriedade e operada por uma única empresa que controla a
maneira como recursos virtualizados e serviços automatizados são customizados e
usados por várias linhas de negócios e grupos constituintes. Nuvens privadas existem
para tirar proveito de muitas eficiências da nuvem, enquanto fornecem mais controle
de recursos e direção sem ocupação variada.

Diferentemente de um data center privado virtual, a infraestrutura utilizada
pertence ao usuário. Portanto, ele possui total controle sobre como as aplicações
são implementadas na nuvem. Uma nuvem privada é, em geral, construída sobre um data
center privado.

Características chave de nuvens privadas incluem:

\subsection{Nuvem pública}

Nuvens públicas são propriedade e operadas por empresas que usam as mesmas para
oferecer acesso rápido a recursos de computação financeiramente suportáveis a outras
organizações e indivíduos. Com serviços de nuvem pública, os usuários não precisam
comprar hardware, software ou infraestrutura de apoio, o que é propriedade e
gerenciado por provedores.

Tecnicamente, pode haver pouca ou nenhuma diferença entre a arquitetura de nuvem
privada e pública. Entretanto, considerações de segurança podem ser substancialmente
diferentes para os serviços disponibilizados para um público e quando a comunicação
é afetada sobre uma rede não confiável.

As aplicações de diversos usuários ficam misturadas nos sistemas de armazenamento,
o que pode parecer ineficiente a princípio. Porém, se a implementação de uma nuvem
pública considera questões fundamentais, como desempenho e segurança, a existência
de outras aplicações sendo executadas na mesma nuvem permanece transparente tanto
para os prestadores de serviços como para os usuários.

As características chaves da nuvem pública são, portanto, segurança e controle
sofisticados projetados para os requisitos específicos de uma empresa.

\begin{figure}[ht]
    \centering
    \includegraphics[width=0.55\textwidth]{img/privatePublic.png}
    \caption{Comparações entre nuvem pública e privada}
\end{figure}

\subsection{Nuvem híbrida}

A nuvem híbrida usa uma base de nuvem privada combinada ao uso estratégico de
serviços de nuvem pública. A realidade é que uma nuvem privada não pode existir
isolada do restante dos recursos de TI e a nuvem pública de uma empresa. A maioria
das empresas com nuvens privadas se desenvolverão para gerenciar cargas de trabalho
entre data centers, nuvens privadas e nuvens públicas - criando, assim, nuvens
híbridas.

Nas nuvens híbridas, há uma composição dos modelos de nuvens públicas e privadas.
Essa característica possui a vantagem de manter os níveis de serviço mesmo que
haja flutuações rápidas na necessidade dos recursos. A conexão entre as nuvens
pública e privada pode ser usada até mesmo em tarefas periódicas que são mais
facilmente implementadas nas nuvens públicas, por exemplo.

O fator chave para o sucesso da nuvem híbrida: a capacidade para gerenciar de
forma eficiente e segura a combinação de serviços de nuvem pública e privada
como um único ambiente de computação unificado, tirando proveito integral da
nuvem.

\subsection{Nuvem comunitária}

Na nuvem comunitária, a infraestrutura de nuvem é compartilhada por diversas
organizações e suporta uma comunidade específica que partilha preocupações
(por exemplo, a missão, os requisitos de segurança, política e considerações sobre
o cumprimento). Pode ser administrado por organizações ou por um terceiro e pode
existir localmente ou remotamente.

% \begin{figure}[ht]
%     \centering
%     \includegraphics[width=0.65\textwidth]{img/modelosNuvem.png}
%     \caption{Características principais de cada modelo}
% \end{figure}
