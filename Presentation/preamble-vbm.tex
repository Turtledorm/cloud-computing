% language and encoding
\usepackage[utf8]{inputenc}
\usepackage[T1]{fontenc}
\usepackage[brazil]{babel}

% fonts: normal and monospaced
%\usepackage[charter]{mathdesign}
\usepackage{mathpazo}
%\usepackage{newtxtext, newtxmath}
%\usepackage[scaled]{ulgothic}
\usepackage{inconsolata}


% additional commands to math mode
\usepackage{amsmath}

% indent first line of first paragraph
\usepackage{indentfirst}

% page numbering
\pagestyle{empty}

% automatically format quotes (no need to write as ``'')
\usepackage{csquotes}
\MakeOuterQuote{"}


% support coloured links, PDF bookmarks, etc.
%\usepackage[colorlinks=false, pdfstartview=FitH, pdfpagelayout=OneColumn]{hyperref}

% remove space after a comma when it acts like a decimal separator
\usepackage{icomma}

% "cancel" terms with a slash
%\usepackage{cancel}

% enable use of colours
%\usepackage{xcolor}

% show customised enumerate 
%\usepackage{enumerate}

% support graphics
\usepackage{graphicx}

% input raw text
%\usepackage{verbatim}

% add cells that span more than one row or column
%\usepackage{multirow, multicol}

% include external raw PDF pages
%\usepackage{pdfpages}

% allow forcing force a figure to be displayed "here"
%\usepackage{float}

% input code
%\usepackage{listings} \lstset{basicstyle=\ttfamily}

% add "lorem ipsum" text
%\usepackage{blindtext} \blindmathtrue

% add questions and answers
\usepackage[]{exercise}
\renewcommand{\ExerciseName}{Questão}
\renewcommand{\ExerciseListName}{Q\!\!}
\renewcommand{\AnswerName}{Resposta}
\renewcommand{\AnswerListName}{Resposta}
\renewcommand{\ExePartName}{Item}
\renewcommand{\ExerciseHeader}{\hrule~\par\noindent{\textbf{\large
            \ExerciseName~\ExerciseHeaderNB\ExerciseHeaderTitle
            \ExerciseHeaderOrigin. \medskip}}}
\renewcommand{\ExePartHeader}{~\par\noindent{\textit{\large
            (\ExePartHeaderNB)
            \smallskip}}}
\renewcommand{\theExePart}{\alph{ExePart}}
\renewcommand{\ExePartHeaderNB}{\theExePart}



% add Portuguese trig functions
\DeclareMathOperator{\sen}{sen}
\DeclareMathOperator{\tg}{tg}
\DeclareMathOperator{\cotg}{cotg}
\DeclareMathOperator{\cossec}{cossec}
\DeclareMathOperator{\arcsen}{arcsen}
\DeclareMathOperator{\arctg}{arctg}
\DeclareMathOperator{\arccotg}{arccotg}
\DeclareMathOperator{\arccossec}{arccossec}

% add British commands and environments
\newenvironment{centre}[0]{\begin{center}}{\end{center}}
\newenvironment{itemise}[0]{\begin{itemize}}{\end{itemize}}
