\section{Gerenciamento da segurança da informação na nuvem}

Para a segurança de uma rede em nuvem, devem-se sempre seguir os princípios abaixo:

\newcommand{\itemm}[1]{\item\textbf{#1}}

\begin{itemise}

    \itemm{Acesso privilegiado de usuários}: A sensibilidade de informações
    confidenciais nas empresas leva a um controle de acesso dos usuários e
    informação bem específica de quem terá privilégio de administrador.
    
    \itemm{Conformidade com regulamentação}: As empresas são responsáveis pela
    segurança, integridade e a confidencialidade de seus próprios dados. Os
    fornecedores de computação em nuvem devem estar preparados para auditorias
    externas e certificações de segurança.
    
    \itemm{Localização dos dados}: A empresa que usa uma nuvem provavelmente não
    sabe exatamente onde os dados estão armazenados, talvez nem o país onde as
    informações estão guardadas. O fornecedor deve estar disposto a se comprometer a
    armazenar e a processar dados em jurisdições específicas, assumindo um
    compromisso em contrato de obedecer os requisitos de privacidade que o país de
    origem da empresa pede.
    
    \itemm{Segregação dos dados}: Geralmente uma empresa divide um ambiente com
    dados de diversos clientes. Com isso, surge a necessidade de separação de dados,
    aplicando-se criptografia.
    
    \itemm{Recuperação dos dados}: O fornecedor da nuvem deve saber onde estão os
    dados da empresa e o que acontece para recuperação de dados em caso de
    catástrofe. Qualquer aplicação que não replica os dados e a infraestrutura em 
    diversas localidades está vulnerável para falha completa. Torna-se importante
    ter um plano de recuperação e um tempo estimado para tal.
    
    \itemm{Apoio à investigação}: A existência de atividades ilegais pode se tornar
    impossível na computação em nuvem, uma vez que há uma variação de servidores
    onde estão localizados os acessos e os dados dos usuários conforme o tempo.
    
    \itemm{Viabilidade em longo prazo}: No mundo ideal, o fornecedor de computação
    em nuvem jamais vai falir ou ser adquirido por uma empresa maior. A empresa
    precisa garantir que os seus dados estarão disponíveis caso o fornecedor de
    computação em nuvem deixe de existir ou seja migrado para uma empresa maior.
\end{itemise}
\undef\itemm

De forma a diminuir o impacto de falhas na segurança, devem-se sempre considerar os
possíveis riscos:
\begin{itemise}
    \item Impacto prejudicial advindo do manuseio inadequado de dados.
    \item Encargos por serviços não autorizados.
    \item Problemas financeiros ou legais do fornecedor.
    \item Problemas operacionais ou encerramentos do fornecedor.
    \item Problemas de recuperação de dados e confidencialidade.
    \item Preocupações gerais com segurança.
    \item Ataques de sistema por forças externas.
\end{itemise}

Com o uso de sistemas na nuvem, há o risco sempre presente da conectividade,
segurança de dados e ações dolosas interferindo com os processos de computação.
Entretanto, com um plano bem pensado, uma metodologia para selecionar o provedor
de serviço e uma perspectiva astuta do gerenciamento de risco em geral, a maioria
das empresas pode usar essa tecnologia com segurança. 
