\section{Topologia: Computação como uma mercadoria}

O conceito da nuvem é construído sobre camadas, cada uma fornecendo um nível
distinto de funcionalidade. Essa estratificação dos componentes forneceu o
meio para que as camadas da computação em nuvem se tornassem uma mercadoria, como
eletricidade, serviço telefônico ou gás natural. A mercadoria que a computação em
nuvem vende é poder computacional a um custo e despesas menores para o usuário.
Espera-se que a computação em nuvem se torne o próximo serviço megautilitário.

Como exemplo, explicaremos o \emph{Virtual Machine Monitor} (VMM) da IBM. Ele fornece o meio para
uso simultâneo das instalações de nuvem (figura \ref{fig:vmm}). VMM é um programa em
um sistema host que permite que um computador suporte diversos ambientes de execução
idênticos. Do ponto de vista do usuário, o sistema é um computador autocontido que
é isolado dos outros usuários. Na realidade, cada usuário está sendo servido pela
mesma máquina. Na computação em nuvem, o VMM permite que usuários monitorem e gerenciem
aspectos do processo, tais como acesso a dados, armazenamento de dados, criptografia,
endereçamento, topologia e movimento de carga de trabalho. 


\begin{figure}[ht]
    \centering
    \includegraphics[width=0.6\textwidth]{img/vmm.png}
    \caption{Como o Virtual Machine Monitor funciona~\cite{cloud-computing-fundamentals}.}
    \label{fig:vmm}
\end{figure}


As camadas principais que a nuvem oferece são~\cite{cloud-computing-fundamentals}:

\newcommand{\itemm}[1]{\item\textbf{#1}}

\begin{itemise}

    \itemm{IaaS --- Infrastructure as a Service:} Camada que constitui a base da nuvem.
    Consiste nos ativos físicos --- servidores, dispositivos de rede, discos de armazenamento,
    etc. Ao usar IaaS, o cliente não controla de fato a infraestrutura subjacente, mas sim
    os sistemas operacionais, armazenamento, aplicativos de implementação e, até certo ponto,
    os componentes de rede selecionados. 

    Serviços de \emph{Print On Demand} (POD) são um exemplo de organizações que podem se
    beneficiar da IaaS. O modelo de POD é baseado na venda de produtos customizados, onde
    pessoas podem abrir lojas e vender designs de produtos. Os lojistas podem carregar
    o número de designs que quiserem à medida que os criam. Muitos carregam milhares.
    Com recursos de armazenamento em nuvem, um POD pode fornecer espaço de armazenamento
    ilimitado.

    \itemm{PaaS --- Plataform as a Service:} Camada intermediária. Fornece acesso a
    sistemas operacionais e serviços associados, além de uma maneira de implementar
    aplicativos para a nuvem usando linguagens de programação e ferramentas suportadas
    pelo fornecedor. Não é necessário gerenciar ou controlar a infraestrutura subjacente;
    o cliente tem controle dos aplicativos implementados e, até certo ponto, de configurações
    de ambiente de \emph{hosting} de aplicativos. 

    PaaS tem provedores como \emph{Elastic Compute Cloud} (EC2) da Amazon. A pequena 
empresa de software é um empreendimento ideal para PaaS. Com a plataforma elaborada,
produtos de classe mundial podem ser criados sem o gasto adicional da produção
interna.

    \itemm{SaaS --- Software as a Service:} É a camada superior (a de aplicativo), a qual
    a maioria visualiza como a nuvem. Aplicativos são executados aqui e são fornecidos
    sob demanda para os usuários. SaaS tem provedores como \emph{Google Pack}, que
    inclui aplicativos que podem ser acessados pela Internet, ferramentas como Gmail,
    Google Talk, Docs e outros.

\end{itemise}

\begin{figure}[H]
    \centering
    \includegraphics[width=0.5\textwidth]{img/services2.png}
    \caption{Principais camadas de computação em nuvem integradas nos componentes
        "como serviço"~\cite{cloud-computing-fundamentals}
    }
    \label{fig:layers}
\end{figure}


Existem também outros serviços oferecidos pela computação em 
nuvem~\cite{informatica}:

\begin{itemise}

    \itemm{DevaaS - Development as a Service:} As ferramentas de desenvolvimento tomam
    forma na computação em nuvem como ferramentas compartilhadas, ferramentas de
    desenvolvimento \emph{web-based} e serviços baseados em \emph{mashup}. 

    \itemm{CaaS - Communication as a Service:} Uso de uma solução de comunicação unificada
    hospedada em data center do provedor ou fabricante (exemplo: Microsoft Lync). 

    \itemm{EaaS - Everything as a Service:} Quando tudo é utilizado: infraestrurura,
    plataformas, software, suporte... Enfim, o que envolve Tecnologia da Informação e
    Comunicação como um serviço. 

    \itemm{DBaas - Database as a Service:} Quando a parte de servidores de banco de dados
    é utilizada como serviço.

    \itemm{TaaS - Testing as a Service:}: Oferece um ambiente apropriado para que o usuário
    possa testar aplicações e sistemas de maneira remota, simulando o comportamento destes
    em nível de execução.

\end{itemise}

\undef\itemm

% \begin{figure}[ht]
%     \centering
%     \includegraphics[width=0.8\textwidth]{img/googlecloud.png}
%     \caption{Exemplo do provedor
%              \href{https://cloud.google.com/}{Google~Cloud~Platform}
%              para os serviços oferecidos para computação em nuvem
%             }
%     \label{img:googlecloud}
% \end{figure}
