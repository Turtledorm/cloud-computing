\section{Introdução}

O termo computação em nuvem, do inglês \emph{cloud computing}, refere-se à
utilização da memória e das capacidades de armazenamento e cálculo de
computadores e servidores compartilhados e interligados por meio da Internet,
seguindo o princípio da computação em grade (modelo computacional capaz de
alcançar uma alta taxa de processamento dividindo as tarefas entre diversas
máquinas, podendo ser em rede local ou rede de longa distância, que formam uma
máquina virtual).

O armazenamento de dados é feito em serviços que podem ser acessados de qualquer
lugar do mundo, a qualquer hora, não havendo necessidade de instalar
programas ou de armazenar dados. O acesso a programas, serviços e arquivos é
remoto, através da Internet --- daí a alusão à nuvem. O uso desse modelo (ambiente)
é mais viável do que o uso de unidades físicas.

Num sistema operacional disponível na Internet, a partir de qualquer computador e em
qualquer lugar, pode-se ter acesso a informações, arquivos e programas num sistema
único, independente de plataforma. O requisito mínimo é um computador compatível
com os recursos disponíveis na Internet.

Na prática, a computação em nuvem seria a transformação dos sistemas
computacionais físicos de hoje em uma base virtual. Ela tem como principal
característica a transformação do tradicional modo de se utilizar e adquirir os
<<<<<<< HEAD
recursos da TI pelas empresas. O processo todo resulta de uma longa transição da
computação baseada em hardware para a computação baseada em software e agora na
Web. As nuvens permitem que recursos não utilizados de computação sejam
compartilhados, ou "virtualizados", e reutilizados por outros clientes, o que
maximiza a eficiência. E para reduzir ainda mais os custos, a maioria desses
sistemas funciona com software de código aberto, de baixo custo.

Segundo estimativas do grupo americano de informática IBM \cite{IBM-cloud-computing},
o mercado mundial de computação em nuvem poderia alcançar os 200 bilhões de dólares
em 2020.
=======
recursos da TI (\emph{Tecnologia da Informação}) pelas empresas. O processo todo
resulta de uma longa transição da computação baseada em hardware para a computação
baseada em software e agora na Web. As nuvens permitem que recursos não utilizados
de computação sejam compartilhados, ou "virtualizados", e reutilizados por outros
clientes, o que maximiza a eficiência. E para reduzir ainda mais os custos, a maioria
desses sistemas funciona com software de código aberto, de baixo custo.

Já existem alguns serviços que, de certa forma, encaixam-se dentro do
conceito de computação em nuvem. Abaixo estão alguns exemplos:

\begin{itemise}
    \item E-mail (Gmail e Yahoo! Mail)

    \item Discos virtuais (Dropbox ou OneDrive)

    \item Armazenamento e compartilhamento de fotos ou vídeos (Flickr e YouTube)
\end{itemise}

% Parece... Coisa de conclusão?
% Segundo estimativas do grupo americano de informática IBM, o mercado mundial de
% computação em nuvem poderia alcançar os 200 bilhões de dólares em 2020.
>>>>>>> 3f5b2060e67de9753da9f41fe9c9dfd30b8b5eb8
