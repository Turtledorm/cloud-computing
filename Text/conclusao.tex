\section{Conclusão}

O termo \textbf{computação em nuvem} é relativamente recente, mas se analisarmos bem,
veremos que a ideia não é necessariamente nova. Serviços de e-mail, como Gmail e
Yahoo! Mail; "discos virtuais" na Internet, como Dropbox ou OneDrive; sites de
armazenamento e compartilhamento de fotos ou vídeos, como Flickr e YouTube. Todos são
exemplos de recursos que, de certa forma, estão dentro do conceito de computação em
nuvem.

Note que todos os serviços mencionados acima não são executados no computador do
usuário, mas este pode acessá-los de qualquer lugar, muitas vezes sem pagar licenças
de software. No máximo, paga-se um valor periódico pelo uso do serviço ou pela
contratação de recursos adicionais, como maior capacidade de armazenamento de dados,
por exemplo.

Ao transferir o modelo atual para a computação em nuvem, empresas e líderes da área 
de TI em todo o mundo estão descobrindo que a adoção da nuvem é mais complicada do 
que o previsto, conforme estudo~\cite{kmpg-cloud-takes-shape} realizado pela KPMG 
International. Segundo o relatório "A nuvem toma forma", aproximadamente 33\% dos 
executivos ouvidos dizem que os custos de implementação foram mais altos do que o 
esperado (ao contrário do que alegam os defensores da computação em nuvem). Uma 
porcentagem igual afirma que a integração dos serviços de nuvem à infraestrutura de 
tecnologia existente foi especialmente difícil.

Para os pequenos usuários, a tendência é que os computadores do futuro terão preços
baixos, se comparados com os preços atuais, visto que serão produzidos de forma mais
simplificada, já que boa parte de seus dados poderão ser armazenados fora do
hardware. Há destaque para os mini laptops, pois podem ser facilmente carregados,
possuem baixo consumo de energia e são bem mais baratos.

Nessa nova e revolucionária era, a computação em nuvem pode fornecer para as 
organizações os meios e métodos necessários para assegurar a estabilidade financeira 
e serviços de alta qualidade~\cite{cloud-computing-fundamentals}. Obviamente, é 
preciso haver cooperação global para que o processo de computação em nuvem atinja 
segurança otimizada e padrões operacionais 
gerais~\cite{cloud-computing-fundamentals}. 
