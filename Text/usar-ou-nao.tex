\section{Usar ou não usar a nuvem?}

As principais preocupações expressas por aqueles que estão adotando a nuvem são
segurança e privacidade. As empresas que fornecem serviços de computação em nuvem
sabem disso e entendem que, sem segurança confiável, seus negócios irão
fracassar. Portanto a segurança e a privacidade são altas prioridades para todas as
entidades de computação em nuvem.


\subsection{Requisitos de largura da banda}

Ao adotar a estrutura de nuvem, a largura da banda e seu potencial gargalo devem ser
considerados na estratégia. No artigo de CIO.com \emph{The Skinny Straw: Cloud
Computing's Bottleneck and How to Address It}, encontra-se a seguinte declaração:

\begin{displayquote}
\emph{
    "Implementadores de virtualização descobriram que o principal gargalo para a 
    densidade de máquinas virtuais é a capacidade de memória. Agora, há um novo
    leque de servidores sendo lançados com áreas de cobertura da memória muito
    maiores, eliminando a memória como um gargalo do sistema. A computação em nuvem
    elimina esse gargalo removendo a questão da densidade de máquinas —- lidar com
    isso é responsabilidade do provedor da nuvem, fazendo com que o usuário não tenha
    com que se preocupar.
    Para a computação em nuvem, a largura da banda, de e para o provedor de nuvem,
    é um gargalo."
}
\end{displayquote}


\subsection{Impacto financeiro}

Como uma proporção considerável do custo em operações de TI vem de funções
administrativas e de gerenciamento, a automação implícita de algumas dessas
funções irá, por si só, cortar custos em um ambiente de computação em nuvem.
A automação pode reduzir o fator de erro e o custo da redundância de repetição
manual significativamente. 

Há outros fatores que contribuem para problemas financeiros, como o custo de
manutenção de instalações físicas, uso de energia elétrica, sistemas de
resfriamento e, obviamente, fatores de administração e gerenciamento. Como se
pode ver, a largura de banda não está sozinha, definitivamente.
