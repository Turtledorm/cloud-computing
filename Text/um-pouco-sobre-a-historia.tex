\chapter{Um pouco sobre a história da computação em nuvem}

Computação em nuvem não é um conceito claramente definido. Não estamos tratando de
uma tecnologia pronta que saiu dos laboratórios pelas mãos de um grupo de
pesquisadores e posteriormente foi disponibilizada no mercado. Essa característica
faz com que seja difícil identificar com precisão a sua origem. Mas há alguns
indícios bastante interessantes.

Um deles remete ao trabalho desenvolvido por John McCarthy. Falecido em outubro de
2011, o pesquisador foi um dos principais nomes por trás da criação do que
conhecemos como inteligência artificial, com destaque para a linguagem Lisp, até
hoje aplicada em projetos que utilizam tal conceito.

\begin{figure}[ht]
    \centering
    \includegraphics[width=0.3\textwidth]{img/mccarthy.jpg}
    \caption{John McCarthy}
    \label{fig:mccarthy}
\end{figure}

Além desse trabalho, John McCarthy tratou de uma ideia bastante importante no início
da década de 1960: computação por tempo compartilhado (\emph{time sharing}), em que um
computador pode ser utilizado simultaneamente por dois ou mais usuários para a
realização de determinadas tarefas, aproveitando especialmente o intervalo de tempo
ocioso entre cada processo.

Quase que na mesma época, o físico Joseph Carl Robnett Licklider entrou para a
história ao ser um dos pioneiros da Internet. Licklider acabou sendo um dos
primeiros a entender que os computadores poderiam ser usados de maneira conectada,
de forma a permitir comunicação de maneira global e, consequentemente, o
compartilhamento de dados. 

Embora possamos associar várias tecnologias, conceitos e pesquisadores ao assunto,
ao juntarmos os trabalhos de John McCarthy e J. C. R. Licklider podemos ter uma
grande ajuda na tarefa de compreender a origem e a evolução da computação em nuvem.

O termo computação em nuvem é muito recente, e surgiu por volta de 2006, em uma
palestra de Eric Schmidt, da Google, descrevendo como sua empresa gerenciava seus
próprios data centers. 

Alguns meses depois é que o termo \emph{cloud} (nuvem) tornou-se mais popular, quando
a Amazon anunciou sua oferta de EC2 (Elastic Computing Cloud).

A computação em nuvem, até poucos anos atrás, era tida como uma tendência, mas hoje
em dia é possível encontrar cada vez mais serviços que funcionam a partir de uma
conexão com a Internet.

Agora, fica a expectativa da evolução da computação em nuvem: por exemplo, será
mesmo possível rodar todo um sistema operacional na nuvem? 
