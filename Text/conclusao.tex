\section{Conclusão}

Ao transferir o modelo atual para a computação em nuvem, empresas e líderes da área 
de TI em todo o mundo estão descobrindo que a adoção da nuvem é mais complicada do 
que o previsto, conforme estudo~\cite{kmpg-cloud-takes-shape} realizado pela KPMG 
International. Segundo este mesmo estudo, aproximadamente 33\% dos executivos ouvidos
dizem que os custos de implementação foram mais altos do que o esperado (ao contrário
do que alegam os defensores da computação em nuvem). Uma porcentagem igual afirma que
a integração dos serviços de nuvem à infraestrutura de tecnologia existente foi
especialmente difícil.

Para os pequenos usuários, a tendência é que os computadores do futuro terão preços
baixos, se comparados com os preços atuais, visto que serão produzidos de forma mais
simplificada, já que boa parte de seus dados poderão ser armazenados fora do
hardware. Há destaque para os mini laptops, pois podem ser facilmente carregados,
possuem baixo consumo de energia e são bem mais baratos.

A computação em nuvem, até poucos anos atrás, era tida como uma tendência, mas hoje
em dia é possível encontrar cada vez mais serviços que funcionam a partir de uma
conexão com a Internet. Agora, fica a expectativa da evolução da computação em nuvem:
por exemplo, será mesmo possível rodar todo um sistema operacional na nuvem? 

Nessa nova e revolucionária era, a computação em nuvem pode fornecer para as 
organizações os meios e métodos necessários para assegurar a estabilidade financeira 
e serviços de alta qualidade~\cite{cloud-computing-fundamentals}. Obviamente, é 
preciso haver cooperação global para que o processo de computação em nuvem atinja 
segurança otimizada e padrões operacionais gerais~\cite{cloud-computing-fundamentals}. 
