\section{Desvantagens em se usar a nuvem}

As principais preocupações expressas por aqueles que estão adotando a nuvem são
segurança e privacidade. As empresas que fornecem serviços de computação em nuvem
sabem disso e entendem que, sem segurança confiável, seus negócios irão
fracassar.

\subsection{Requisitos de largura da banda}

Ao adotar a estrutura de nuvem, a largura da banda e seu potencial gargalo devem ser
considerados na estratégia. No artigo de CIO.com \emph{The Skinny Straw: Cloud
Computing's Bottleneck and How to Address It}, encontra-se a seguinte declaração:

\begin{displayquote}
\emph{
    "Implementadores de virtualização descobriram que o principal gargalo para a 
    densidade de máquinas virtuais é a capacidade de memória. Agora, há um novo
    leque de servidores sendo lançados com áreas de cobertura da memória muito
    maiores, eliminando a memória como um gargalo do sistema. A computação em nuvem
    elimina esse gargalo removendo a questão da densidade de máquinas --- lidar com
    isso é responsabilidade do provedor da nuvem, fazendo com que o usuário não tenha
    com que se preocupar.
    Para a computação em nuvem, a largura da banda, de e para o provedor de nuvem,
    é um gargalo."
}
\end{displayquote}

\subsection{Problemas específicos}

Infelizmente, a nuvem também traz problemas para algumas empresas usarem-na:

\newcommand{\itemm}[1]{\item\textbf{#1}}

\begin{itemise}

    \itemm{Tempo gasto para realizar a transição:} Uma empresa não pode migrar
    diretamente suas aplicações atuais para aplicações em nuvem. Isto requer
    personalização e tempo gasto em mover dados valiosos para aplicações mais
    amigáveis.

    \itemm{Surgimento de dificuldades durante a mudança:} O departamento de TI
    precisa realizar a transição de monitoramento de hardware para monitorar o uso
    de software. Tal procedimento pode levar a múltiplas alterações nas ferramentas de
    gerenciamento, além da preocupação em garantir a estabilidade e acessibilidade
    da nuvem a qualquer momento.

\end{itemise}

\undef\itemm
