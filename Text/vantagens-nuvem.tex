\section{Vantagens em se usar a nuvem}

Há razões válidas e significativas, de negócios e de TI, para as empresas transferirem
seus dados para trabalhar na nuvem. Os aspectos fundamentais são:

\newcommand{\itemm}[1]{\item\textbf{#1}}

\begin{itemise}
    \itemm{Redução de custo:} A computação em nuvem pode reduzir os custos de
        despesas de capital e despesas operacionais, pois os recursos só são adquiridos
        quando necessário, e só se paga por eles quando são usados. Ou seja, a necessidade
        de se investir em infrastrutura é menor.

    \itemm{Uso refinado da equipe:} Usar a computação em nuvem libera equipe de
        valor, permitindo que eles se concentrem em entregar valor, e não em manter
        hardware e software.

    \itemm {Escalabilidade robusta:} A computação em nuvem permite escala imediata,
        para mais ou para menos, a qualquer momento, sem compromisso a longo prazo.
        Por exemplo: se houvesse necessidade de ampliar um serviço de uma empresa, talvez
        fosse necessário comprar mais hardware. Com a computação em nuvem, essa escala
        é facilmente e rapidamente ajustada; basta adquirir mais espaço na nuvem.

\end{itemise}

Outras vantagens incluem maior segurança (requisito implementado por diversas companhias que
oferecem serviços de nuvem), organização (é possível controlar quais trabalhadores têm acesso
a quais arquivos) e flexibilidade de acesso (a nuvem pode ser acessada de qualquer lugar).

\undef\itemm
