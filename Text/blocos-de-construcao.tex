\chapter{Blocos de construção da computação em nuvem}

\newcommand{\frontend}{\emph{front-end}\xspace}
\newcommand{\backend} {\emph{back-end}\xspace}

O modelo de computação em nuvem é composto de um \frontend e um \backend. Esses dois
elementos são conectados por meio de uma rede, geralmente a Internet. O \frontend é
o veículo pelo qual o usuário interage com o sistema; o \backend é a própria nuvem.
O \frontend é composto de um cliente de computador, ou a rede de computadores de um
empreendimento, e os aplicativos usados para acessar a nuvem. O \backend fornece os
aplicativos, computadores, servidores e armazenamento de dados que criam a nuvem de
serviços. 
